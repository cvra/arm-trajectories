\documentclass[11pt,onecolumn,a4paper,oneside,openright,final]{article}
\usepackage{amsmath}
\setlength{\parindent}{0pt}

\begin{document}

We have a spline $\mathbf{s}\left( r \right)$ which we reparametrize to $\mathbf{s}\left( r \left( \sigma \right) \right)$ such that $\left| \left| \frac{\partial \mathbf{s}}{\partial \sigma} \right| \right| = 1$.
We're interested in calculating the maximal possible speed along the spline.

The reparametrization is being done by solving the integral (\ref{equ:reparam_int}) for $r$.
\begin{equation}
    \sigma \left( r \right) = \int_0^r{\left| \left| \dot{\mathbf{s}} \left( \xi \right) \right| \right| d\xi}    
    \label{equ:reparam_int}
\end{equation}
This has to be done numerically.


Derivative in respect to time:

\begin{equation}
    \frac{\partial}{\partial t} \mathbf{s} \left( \sigma \left( t \right) \right) =
    \frac{\partial \mathbf{s}}{\partial \sigma} \frac{\partial \sigma}{\partial t} 
    \label{equ:s_dot}
\end{equation}

\begin{equation}
    \frac{\partial^2}{\partial t^2} \mathbf{s} \left( \sigma \left( t \right) \right) =
    \frac{\partial^2 \mathbf{s}}{\partial \sigma^2} \frac{\partial \sigma}{\partial t} +
    \frac{\partial \mathbf{s}}{\partial \sigma} \frac{\partial^2 \sigma}{\partial t^2} 
    \label{equ:s_dot_dot}
\end{equation}

We consider the case of no acceleration along the reparametrization ($\frac{\partial^2 \sigma}{\partial t^2} = 0$) to calculate the highes possible speed at that point.
Thus we end up with
\begin{equation}
    \frac{\partial^2}{\partial t^2} \mathbf{s} \left( \sigma \left( t \right) \right) =
    \frac{\partial^2 \mathbf{s}}{\partial \sigma^2} \frac{\partial \sigma}{\partial t}
    \label{equ:s_dot_dot_no_acc}
\end{equation}

Considering that we want the maximal rate at which we can change $\sigma$ over time without having higher accelerations than each actuator can handle, we get
\begin{equation}
    \frac{\partial \sigma}{\partial t} =
    \min_i{\frac{a_{\max,i}}{\left|\frac{\partial^2 s_i}{\partial \sigma^2}\right|}}
    \label{equ:max_speed}
\end{equation}

The value of $\frac{\partial^2 \mathbf{s}}{\partial \sigma^2}$ has still to be calculated.

The derivative of the spline in respect to the reparametrization $\sigma$:
\begin{equation}
    \frac{\partial}{\partial \sigma} \mathbf{s} \left( r \left( \sigma \right) \right) =
    \frac{\partial \mathbf{s}}{\partial r} \frac{\partial r}{\partial \sigma}
    \textrm{ with }
    \left| \left| \frac{\partial}{\partial \sigma} \mathbf{s} \left( r \left( \sigma \right) \right) \right| \right| = 1
    \label{equ:s_d_sigma}
\end{equation}

Further,
\begin{equation}
    \left| \left| \frac{\partial \mathbf{s}}{\partial r} \frac{\partial r}{\partial \sigma} \right| \right| = 
    \left| \left| \frac{\partial \mathbf{s}}{\partial r} \right| \right| \cdot \left| \frac{\partial r}{\partial \sigma} \right| = 1
    \label{equ:s_d_sigma_cont}
\end{equation}

\begin{equation}
    \left| \frac{\partial r}{\partial \sigma} \right| =
    \frac{1}{\left| \left| \frac{\partial \mathbf{s}}{\partial r} \right| \right|}
    \label{equ:s_d_sigma_cont_2}
\end{equation}


The second derivative:
\begin{equation}
    \frac{\partial^2}{\partial \sigma^2} \mathbf{s} \left( r \left( \sigma \right) \right) =
    \frac{\partial^2 \mathbf{s}}{\partial r^2} \frac{\partial r}{\partial \sigma} +
    \frac{\partial \mathbf{s}}{\partial r} \frac{\partial^2 r}{\partial \sigma^2} 
    \label{equ:s_dd_sigma}
\end{equation}

From (\ref{equ:reparam_int}) we get
\begin{equation}
    \frac{d \sigma}{d r} = \left| \left| \dot{\mathbf{s}} \left( r \right) \right| \right|
    \label{equ:d_sigma_d_r}
\end{equation}

which leads to

\begin{equation}
    \frac{d r}{d \sigma} = \frac{1}{\left| \left| \dot{\mathbf{s}} \left( r \right) \right| \right|}
    \label{equ:d_r_d_sigma}
\end{equation}
and is consistent with (\ref{equ:s_d_sigma_cont_2}).

We know everything except $\frac{\partial^2 r}{\partial \sigma^2}$. Because we don't accelerate along the spline (see equation~\ref{equ:s_d_sigma}), we can say that

\begin{equation}
    \frac{\partial \mathbf{s}}{\partial \sigma} \bullet \frac{\partial^2 \mathbf{s}}{\partial \sigma^2} = 0
    \label{equ:perpendicular}
\end{equation}

which implies that the acceleration is perpendicular to the velocity.

\begin{equation}
    \frac{\partial \mathbf{s}}{\partial \sigma} \bullet \frac{\partial^2 \mathbf{s}}{\partial \sigma^2} =
    \left(
    \frac{\partial \mathbf{s}}{\partial r} \frac{\partial r}{\partial \sigma}
    \right) \bullet \left(
    \frac{\partial^2 \mathbf{s}}{\partial r^2} \frac{\partial r}{\partial \sigma} +
    \frac{\partial \mathbf{s}}{\partial r} \frac{\partial^2 r}{\partial \sigma^2} 
    \right)
    \label{equ:perpendicular_1}
\end{equation}

\begin{equation}
    0 =
    \left( \frac{\partial \mathbf{s}}{\partial r} \bullet \frac{\partial^2 \mathbf{s}}{\partial r^2} \right) {\left( \frac{\partial r}{\partial \sigma} \right)}^2
    + {\left( \frac{\partial \mathbf{s}}{\partial r} \right)}^2 \frac{\partial r}{\partial \sigma} \frac{\partial^2 r}{\partial \sigma^2}
    \label{equ:perpendicular_2}
\end{equation}

with (\ref{equ:s_d_sigma}) and (\ref{equ:s_d_sigma_cont_2}) we can simplify to

\begin{equation}
    0 =
    \left( \frac{\partial \mathbf{s}}{\partial r} \bullet \frac{\partial^2 \mathbf{s}}{\partial r^2} \right)
    \frac{1}{{\left| \left| \frac{\partial \mathbf{s}}{\partial r} \right| \right|}^2}
    + {\left| \left| \frac{\partial \mathbf{s}}{\partial r} \right| \right|} \frac{\partial^2 r}{\partial \sigma^2}
    \label{equ:perpendicular_3}
\end{equation}

because ${\left( \frac{\partial \mathbf{s}}{\partial r} \right)}^2 \frac{\partial r}{\partial \sigma}  = \frac{\partial \mathbf{s}}{\partial r} \bullet \frac{\partial \mathbf{s}}{\partial r} \frac{\partial r}{\partial \sigma}$
and $\frac{\partial \mathbf{s}}{\partial r} \frac{\partial r}{\partial \sigma}$ is a vector of magnitude $1$ in the same direction as $\frac{\partial \mathbf{s}}{\partial r}$.

Solving for $\frac{\partial^2 r}{\partial \sigma^2}$:

\begin{equation}
    \frac{\partial^2 r}{\partial \sigma^2} =
    - \frac{1}{{\left| \left| \frac{\partial \mathbf{s}}{\partial r} \right| \right|}^3}
    \left( \frac{\partial \mathbf{s}}{\partial r} \bullet \frac{\partial^2 \mathbf{s}}{\partial r^2} \right)
    \label{equ:r_dd_sigma_1}
\end{equation}

$ \frac{\partial\mathbf{s}}{\partial r} $ is free lunch as well as $ \frac{\partial^2 \mathbf{s}}{\partial r^2} $.

Putting (\ref{equ:r_dd_sigma_1}) in (\ref{equ:s_dd_sigma}) we get

\begin{equation}
    \frac{\partial^2}{\partial \sigma^2} \mathbf{s} \left( r \left( \sigma \right) \right) =
    \frac{\partial^2 \mathbf{s}}{\partial r^2} \cdot {\left| \left| \frac{\partial \mathbf{s}}{\partial r} \right| \right|}^{-1} -
    \frac{\partial \mathbf{s}}{\partial r} 
    \left( \frac{\partial \mathbf{s}}{\partial r} \bullet \frac{\partial^2 \mathbf{s}}{\partial r^2} \right)
    {\left| \left| \frac{\partial \mathbf{s}}{\partial r} \right| \right|}^{-3} 
    \label{equ:s_dd_sigma}
\end{equation}

\begin{equation}
    \frac{\partial^2}{\partial \sigma^2} \mathbf{s} \left( r \left( \sigma \right) \right) =
    {\left| \left| \frac{\partial \mathbf{s}}{\partial r} \right| \right|}^{-1}
    \left(\frac{\partial^2 \mathbf{s}}{\partial r^2} -
    \underbrace{\frac{\frac{\partial \mathbf{s}}{\partial r}}{{\left| \left| \frac{\partial \mathbf{s}}{\partial r} \right| \right|}}
    \left( \frac{\frac{\partial \mathbf{s}}{\partial r}}{{\left| \left| \frac{\partial \mathbf{s}}{\partial r} \right| \right|}}
    \bullet \frac{\partial^2 \mathbf{s}}{\partial r^2} \right)}_\text{acc. along the spline}
    \right)
    \label{equ:s_dd_sigma_clean}
\end{equation}

The result (\ref{equ:s_dd_sigma_clean}) can be interpreted geometrically as the spline's acceleration without the part along the spline times the factor imposed by the reparametrization (\ref{equ:s_d_sigma_cont_2}).

\end{document}
